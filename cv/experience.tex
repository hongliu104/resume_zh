%-------------------------------------------------------------------------------
%    SECTION TITLE
%-------------------------------------------------------------------------------
\cvsection{工作经历}


%-------------------------------------------------------------------------------
%    CONTENT
%-------------------------------------------------------------------------------
\begin{cventries}

%---------------------------------------------------------
  \cventry
    {高级软件主管工程师\&高级解决方案主管工程师} % Job title
    {燧原智能科技(成都)有限公司} % Organization
    {成都} % Location
    {2021.07 - 至今} % Date(s)
    {
      \begin{itemize}[parsep=0.2ex] % Description(s) of tasks/responsibilities
        \item \descriptionstyle{主力参与自研大模型推理框架研发,完成流水并行与张量并行POC,支持了llama, baichuan, yi, chatglm等多个系列模型;}
        \item \descriptionstyle{带领4人团队研发topgraph builder,为基于mlir的opset自动生成op builder C++/Python API, 支持对接了多个主流框架:pytorch,onnx,tensorflow,paddlepaddle;}
        \item \descriptionstyle{带领2人以图模式对接oneflow,完成ViT(Vision Transformer)训练的POC;}
        \item \descriptionstyle{主导完成大模型llama pytorch原生推理的移植和优化;}
      \end{itemize}
    }

%---------------------------------------------------------
  \cventry
    {深度学习算法\&研发工程师} % Job title
    {北京比特大陆科技有限公司} % Organization
    {北京,成都} % Location
    {2016.10 - 2021.07} % Date(s)
    {
      \begin{itemize}[parsep=0.2ex] % Description(s) of tasks/responsibilities
        \item \descriptionstyle{带领7人团队研发基于深度学习加速芯片-BM1684的上层软件SDK(提供Python/C++ api), 集成了bmlib/bmdecoder/bmcv/bmruntime,支持1684全系列产品:SC5/SM5/SE5。开源地址: https://github.com/sophon-ai-algo/sophon-inference}
        \item \descriptionstyle{独立完成对CV领域的TensorFlow/Pytorch/MXNet/Caffe的主流的分类/检测模型(共计122个), 在SC5H(17.6T INT8算力,2.2T FP32算力)上进行精度(Top5/mAP)速度的全流程QA的测试代码开发、部署、结果收集,发现并推动修复5个模块的17个bug;}
        \item \descriptionstyle{儿童故事书OCR,“小萝卜”儿童陪伴机器人自动读取故事书上文字(EAST + CRNN),并朗读出来(调用外部API);}
        \item \descriptionstyle{优选开源算法移植,开源地址:https://gitee.com/hong\_liu/sophon-algorithm/tree/master/apps/wav2lip;}
      \end{itemize}
    }

%---------------------------------------------------------
  \cventry
    {深度学习算法工程师(T4-T5)} % Job title
    {百度深度学习实验室(IDL)} % Organization
    {北京} % Location
    {2015.12 - 2016.10} % Date(s)
    {
      \begin{itemize}[parsep=0.2ex] % Description(s) of tasks/responsibilities
        \item \descriptionstyle{凤巢query-ads相关性模型,从LSTM到CNN+Fine-tuning, AUC提升4.1到79.2,预测速率从12ms/sample提升到0.2ms/sample;}
        \item \descriptionstyle{百度金融风控模型,基于百度用户画像、搜索、贴吧、糯米、钱包等10万+维度特征数据,GINI为62(AUC=81, 比LR提升6);}
        \item \descriptionstyle{百度外卖出餐时间预测,预测用户下单到商户做好菜品的时间,以优化外卖骑士的调度方案,平均误差由11分钟降到7分钟;}
        \item \descriptionstyle{百度外卖下单概率预测,在不明显减少总订单量的前提下有效减少优惠补贴总额,总流水减少1.87\%时,补贴额度减少9.82\%;}
      \end{itemize}
    }

%---------------------------------------------------------
  \cventry
    {系统研发工程师(T3-T4)} % Job title
    {百度系统部(SYS)} % Organization
    {北京} % Location
    {2014.4 - 2015.12} % Date(s)
    {
      \begin{itemize}[parsep=0.2ex] % Description(s) of tasks/responsibilities
        \item \descriptionstyle{网络流量预测,外网流量预测用于检测攻击流量以减少带宽费用,内网流量预测用于IDC之间流量调度已以实现负载均衡,提出用小波分解提取流量序列的时域与频域特征,并用NN进行训练;}
        \item \descriptionstyle{机房智能散热,预测机房PUE等重要参数,进而给出冷水机组运行模式建议,采用背包算法给出应该开启冷机的台数;}
        \item \descriptionstyle{硬件数据挖掘平台的设计与实现,支持分布式,承载磁盘预警、流量预测、服务器功耗预测等多个在线应用;}
      \end{itemize}
    }

%---------------------------------------------------------
  \cventry
    {软件研发实习生} % Job title
    {西门子中国研究院} % Organization
    {北京} % Location
    {2011.09 - 2012.04} % Date(s)
    {
      \begin{itemize}[parsep=0.2ex] % Description(s) of tasks/responsibilities
        \item \descriptionstyle{参与项目"Cloud-PDMS(基于云存储的个人数据管理系统)",将员工个人数据归档备份到HDFS,恢复到个人电脑;}
      \end{itemize}
    }

%---------------------------------------------------------
\end{cventries}
